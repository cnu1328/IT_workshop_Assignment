% Creating a simple Title Page in Beamer
\documentclass{beamer}

% Theme choice:
%\usetheme{AnnArbor}
%\usetheme{CambridgeUS}
\usetheme{Antibes}

% Title page details: 
% \title{Your First \LaTeX{} Presentation}
% % \subtitle{My-subtitle}
% \author{Srinivas}
% % \institute{Online Beamer Tutorials}

% % Multiple authors
% \author{Srinivas \and Babul Partheiv \and Sai Charan}
% %   Second~Author \and
% %   Third~Author \and
% %   Fourth~Author \and
% %   Fifth~Author}

% % Two authors with different affiliations
% % \author{First Author\inst{1} \and Second Author\inst{2}}
% % \institute{\inst{1} Affiliation of the 1st author \and
% %  \inst{2} Affiliation of the 2nd author}

% \date{}

%\logo{\includegraphics[width=3cm]{logo.png}}

%Logo only on title page
\titlegraphic{\includegraphics[width=2cm]{logo.png}}

%Modify footer text: 
\title[Center texT]{Magic Ball}
\subtitle{Eager to know}
\author[Left text]{Srinivas}
\date[Right text]{\today}

%Define a counter
\newcounter{currentenumi}

\begin{document}

%Title page frame
\begin{frame}
    \titlepage
\end{frame}

% % Outline frame
\begin{frame}{It Listens My Commands}
    \tableofcontents
\end{frame}

% Presentation structure
\section{Gravitation}
\subsection{
We all know that Gravity is filled every corner of the world. So, as we place a object in air, the Earth attracts the object. Right?}
\section{How does a ball stay in air?}
    \subsection{Actually, This ball listens all my commands.}
    \subsubsection{The Commands like Go Down, Slowly Go Down and Stop}
    \subsection{The Action}
    \subsubsection{}
    \subsection{Method 3}
\section{Comparative study}
\section{References}



\begin{frame}{Introduction}
    \begin{block}{Details}
        Name : Srinivas 
        
        Village : Chengal
        District : Nizamabad
    \end{block}
    \begin{alertblock}{Studying}
        
    \end{alertblock}
    \begin{exampleblock}{An example of typesetting tool}
        Example: MS Word, \LaTeX{}
    \end{exampleblock}
\end{frame}


\begin{frame}{My first table}
\begin{tabular}{|c||l||r|}
\hline
    centered & left-aligned & right-aligned \\ 
\hline
    A & C & E\\ 
\hline
    B & D & F\\ 
\hline
\end{tabular}
\end{frame}



\end{document}


