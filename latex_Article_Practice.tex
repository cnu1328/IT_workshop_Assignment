\documentclass{article}
\usepackage[utf8]{inputenc}

\usepackage{amsmath}
\usepackage{mathtools}
\usepackage[talbe,xcdraw]{xcolor}
\usepackage{graphicx}

\title{Latex Learning}
\author{srinudarpally }
\date{21-08-22}

\begin{document}

\maketitle
\tableofcontents

\section{Introduction to Latex}
Briefly put, LaTeX is a text editing and typesetting platform designed espe-
cially for the use in Mathematics, Engineering, Science and Industry. How-
ever, it should be noted that LaTeX is not only restricted to these fields.
1

\subsection{How to Install Latex}
\subsubsection{in windows}
\paragraph{first install MikTex} 
\paragraph{then install any editor like text studio, tex maker, textlive....}

\subsubsection{in Linux}
\paragraph{in ubuntu sudo apt=get install kile}

\section{What is the purpose \LaTeX?}
LaTeX is a tool that can be used to produce professional quality
documents that employ symbolic representation, charts and notation for
quantitative and mathematical ideas and concepts. It can also be used to
compose high quality presentations in power points, publish journal papers
and books.

\section{What are its benefits?}
For the working mathematician, engineer or scientist, LaTeX has a library
of commands which relay standard symbolic and notational conventions
which allow for ease in representing and conveying mathematical
constructs in a clear and easy manner.

\section{TExt fontss Repressentations}

\begin{tiny}
Welcome to RGUKT
\end{tiny} \\

\begin{small}
Welcome to RGUKT
\end{small} \\
\begin{large} 
Welcome to RGUKT
\end{large} \\
\begin{Large}
Welcome to RGUKT
\end{Large} \\
\begin{Huge}
Welcome to RGUKT
\end{Huge} 
\\
\textit{Welcome to RGUKT}
\\
\textbf{Welcome to RGUKT}

\\
\textsc{Welcome to RGUKT}
\\
\texttt{Welcome to RGUKT}


\\
\underline{Welcome to RGUKT}

\\
{\color{red}{Welcome to RGUKT}}
\\

\pagebreak

\huge{Welcome to RGUKT} \\
\Large{Welcome to RGUKT}

\section{Table Representation}

\begin{table}[h]
    \centering
    \begin{tabular}{|l|l|l|}
    \hline
        S.No  & Subject1 & Subject2
         \\
         \hline
         1 & Maths & IT \\
         \hline
         2 & chemistry & Physics \\
         \hline
         
    \end{tabular}
    \caption{This is my first table}
    \label{tab:my_label}
\end{table}

\begin{table}[h]
    \begin{tabular}{|l|l|l|l|l|}
    \hlinet
    S.No.  & s1 & s2 & s3 & Total \\
    \hline
    1 & 90 & 66 & 0 &xx \\
    \hline
    2 & 0 & 66 & 55 &xx \\
    \hline
     3 & 00 & 55 & 5 &xx\\
    \hline
       4  & 00 & 55 & 55 &xx\\
    \hline
    
    \end{tabular}

\end{table}

\section{Maths Equations}
$(a+b)^2 = a^2+b^2+2ab$ \\
$(\frac{a}{b})^2=\frac{a^2}{b^2}$


\section{Images Representation}
\begin{figure}[h]
    \centering
    \includegraphics[scale= 0.5]{k.jpeg}
    \caption{This is my first image}

\end{figure}

\begin{equation}
    (a+b)^2 = a^2+b^2+2ab
\end{equation}

\begin{itemize}
    \item Roll Number 1
    \item Roll Number 2
\end{itemize}

\begin{enumerate}
    \item Roll Number 3
    \item Roll Number 4
\end{enumerate}

\end{document}
